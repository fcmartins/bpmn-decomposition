\documentclass[a4paper]{article}
\usepackage{amsmath}
\title{Interaction with the CSN compiler}
\author{Tiago Cogumbreiro}

\begin{document}

\maketitle

\section{Validation}

For verifying the validity of a CSN source, we define the operation
\emph{validate} of the interface \textsf{IValidator} that expects a
file name and the success of the operation and a list of messages,
ordered by the position each was found in the source code:
\begin{equation*}
  \operatorname{validate}(\mathsf{String}) \rightarrow \mathsf{Result}
\end{equation*}

A result is a structures that consists of:
\begin{description}
\item[is successful] Specifies if the target file name is valid (or any file
  name it itself includes).
\item[messages] A sequence of messages related to \textit{e.g.\@}
  errors, or warnings.
\end{description}
\begin{equation*}
  \mathsf{Result} = \mathsf{boolean}\ is\ successful,\ \mathsf{List(Message)}\ messages
\end{equation*}

A message is a class with the following attributes:
\begin{description}
\item[title] A one line description of the message.
\item[summary] A multi-line description of the message. For
  example, showing the expected type of a certain variable.
\item[file name] A file-system dependent file name, containing the
  source code.
\item[column] The offset, from the start of a line, in terms of
  characters, where the error is found. The base of the index is 1.
\item[line] The offset, from the start of the file, in terms of lines,
  where the error is found. The base of the index is 1.
\item[level] The severity of the error. This specification of this
  datum is specified next.
\end{description}
\begin{equation*}
  \mathsf{Message} = \mathsf{String}\ title, \mathsf{String}\ summary, \mathsf{String}\ filename, \mathsf{int}\ column, \mathsf{int}\ line, \mathsf{Level}\ level
\end{equation*}

The \emph{level} is a class that follows
\texttt{java.util.logging.Level} (we may use this class).

\end{document}
